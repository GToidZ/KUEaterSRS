\chapter{Software Architecture Design}
\label{chap:software-architecture-design}
<TIP: Describe how you design your application using Unified Modelling
Language (UML). There should be at least two diagrams that describe the
software architecture. You may add additional or remove unnecessary diagrams.
However, there needs to be a coherency between them at the end./>

\section{Domain Model}
\label{section:domain-model}
<TIP: Describe the business concept of your project. Showcase a
domain model that captures the said concept./>

\section{Design Class Diagram}
\label{section:design-class-diagram}
<TIP: Showcase a design class diagram for your project and explain
how it works here. You can group classes into packages or layers to communicate your
design better./>

\section{Sequence Diagram}
\label{section:sequence-diagram}
<TIP: Sequence diagrams describe how the software runs at runtime.
You do not have to create a sequence diagram for every scenario. However,
there should be one for all the main ones./>

<ChatGPT: Creating a sequence diagram for every use case is not
strictly necessary, but it can be a valuable tool in certain situations. Sequence
diagrams are particularly useful for illustrating the interactions between different
components or objects in a system over time, showcasing the flow of messages
or actions between them./>

\section{Algorithm}
\label{section:algorithm}
<TIP: Optional, If you are working on a research project that proposes a new
algorithm, you can describe your algorithm here. It can be in the form of
pseudocode or any diagram that you deem appropriate./>