\chapter{Literature Review and Related Work}
\label{chap:relatedworks}

\section{Competitor Analysis}
\label{section:competitor-analysis}

\begin{table}[h!]
    \begin{adjustwidth}{-.85in}{-.85in}
        \noindent
        \centering
        \small\begin{tabularx}{1.3\textwidth}{|X|>{\columncolor{green!20}}X|X|X|X|}
            \hline & \textbf{KU Eater} & \textbf{Wongnai} & \textbf{Retty} & \textbf{OpenRice} \\\hline
            \textbf{Target Audience} & Kasetsart University students and staffs & General public in Thailand & General public in Japan & General public in Southeast Asia \\\hline
            \textbf{Platform} & Web browsers \& Mobile app & Web browsers \& Mobile app & Web browsers \& Mobile app & Web browsers \& Mobile app \\\hline
            \textbf{Personalization} & AI-driven recommendations based on user preferences & Restaurant search \& discovery, user reviews & Curated restaurant lists, user reviews & Restaurant reviews and recommendations \\\hline
            \textbf{Menu Variety} & Focused on Kasetsart University cafeterias' menu; developers provide data & Local eateries, the developers provide data & Local restaurants, the users provide most data & Extensive restaurant listings, the users can provide data \\\hline
            \textbf{Dietary Filters} & Dietary restrictions available & Many filters but not AI-driven & No dietary filters & Filters based on cuisine, broad dish type \\\hline
            \textbf{User Generated Content} & Reviews and photos of food from stalls in cafeteria & User reviews and photos & Can add new locations to platform, reviews and photos & Able to add locations but moderated, reviews, pictures \\\hline
            \textbf{User Reviews} & User review stalls and dishes offered at the stall & User ratings and popularity scores & User ratings based on satisfaction & User ratings and comments \\\hline
            \textbf{User Rating} & Manual rating and calculated from sentimental analysis & User ratings and popularity scores & User ratings based on satisfaction & User ratings and comments \\\hline
            \textbf{Pricing} & Free for all users & Free with gamification features & Free & Free but with paid services \\\hline
        \end{tabularx}
    \end{adjustwidth}
    \caption{Competitor Analyis of KU Eater}
\end{table}

In order to acknowledge the existing technology with similiar functionalities, a competitor analysis is conducted.
There are various applications in the market that can recommend restaurants, we hand picked 3 of which resembles our project most:

\begin{itemize}
    \item \textbf{\textit{Wongnai}}---a restaurant and hotel review application focusing on Thailand's locality.
    \item \textbf{\textit{Retty}}---a user-centric restaurant review application for Japanese.
    \item \textbf{\textit{OpenRice}}---a restaurant review platform focusing on Asian cuisine.
\end{itemize}

\par
KU Eater is an application that solely focuses on the scope of cafeterias in Kasetsart University.
Unlike our competitors, users in our application can state their dietary restrictions such as being a vegetarian,
not liking certain ingredients etc. Those factors are parameters to our AI-driven recommendation system.
KU Eater is also different from other food reviewing platforms because of the depth our application harnesses,
users are able to see the individual dishes information and popularity score.

\section{Literature Review}
\label{section:literature-review}

The integration of AI and ML in personalized food recommendation systems represents a transformative approach to addressing
the diverse dietary needs and preferences within university campuses. Marquis et al. \cite{marquisetal:2018}
delve into the determinants of university students' food behavior, underscoring the necessity of tailoring
food recommendations to individual preferences, a core functionality of the KU Eater application.

Singh and Dwivedi \cite{singhanddwivedi:2023} proposed an approach to food recommendation system using K-nearest neighbor technique.
Their work provides an exemplary foundation for KU Eater to build a customized recommendation system upon,
and is relevant to KU Eater since their method took dietary requirements into account.

Thongthanomkul's \cite{thongthanomkul:2020} article on improving Wongnai's search system with Machine Learning
and Wongnai's \cite{wongnaicorpus:github} corpus provides examples of how artificial intelligence enhances search functionality.

Operational and economic considerations, as evidenced in the successes of large-scale recommendation systems \cite{sangwan:2023,netflix:2022},
provide a blueprint for KU Eater's potential impact. The application not only addresses the immediate needs of campus
dining but also sets a precedent for the role of AI and ML in creating a dynamic and responsive university dining ecosystem.

In conclusion, the reviewed literature shows importance in the role of AI and ML in improving the dining experience on university campuses.
KU Eater, with its emphasis on personalized recommendations and user feedback analysis,
to address the nuanced demands of campus dining, offering a promising solution to the challenges identified in the current dining system.